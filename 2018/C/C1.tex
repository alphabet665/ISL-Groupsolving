% January 2022

\begin{problem}[ISL 2018 C1]
    Let $n\geqslant 3$ be an integer. Prove that there exists a set $S$ of $2n$
    positive integers satisfying the following property: For every $m=2,3,\ldots,n$
    the set $S$ can be partitioned into two subsets with equal sums of elements,
    with one of subsets of cardinality $m$.
\end{problem}

\begin{solution}[Ritwin]
    We first give an example of a solution for $n=3$:
    \[S = \{1, 3, 4, 5, 6, 7\}\]
    For $m=2$, we can split it up into $\{1,3,4,5\} \cup \{6,7\}$\\
    For $m=3$, we can split it up into $\{1,5,7\} \cup \{3,4,6\}$
    
    Thus, now we will have $n \geq 4$.
    
    Consider \[S = \{G_1, G_2, G_3, \ldots, G_{2n-1}, X\}\] where $G_k$ denotes
    the $k^\textrm{th}$ term of our Fibonacci-like sequence, defined as
    $G_1 = 69$, $G_2 = 420$, $G_k = G_{k-1} + G_{k-2}$ and \[X = \sum_{k=1}^{2n-4} G_k.\]
    
    Note that \[\sum_{j=1}^k G_j = G_{k+2} - G_2\] (which can be proven easily
    by induction), so $X = G_{2n-2} - G_2$. Note that for $n \geq 4$,
    $G_{2n-2} - G_{2n-3} = G_{2n-4} > G_2$ because $2n-4 > 2$ and $G_k$ is a
    strictly increasing sequence, so our set $S$ does not contain any duplicates
    (and has size $2n$, as desired).
    
    Now, we will prove that this works. Let $A$ and $B$ be the two subsets we split $S$ into.
    
    Consider the $m=2$ case: We split $S$ into $A = \{G_1, G_2, G_3, \ldots G_{2n-2}\}$
    and $B = \{G_{2n-1}, X\}$ (so $S = A \cup B$ and $A \cap B = \{\}$), and
    because $G_{2n-1} = G_{2n-2} + G_{2n-3}$ and our definition of $X$, we can
    see that this works, and satisfies $m = |B| = 2$.
    
    Now, for $m > 2$, we can simply take the solution for $m-1$, and alter it a
    bit. Take $G_k$ from $B$, where $k$ is as small as possible. Now, move it to
    $A$, and move $G_{k-1}$ and $G_{k-2}$ from $A$ to $B$. We can see that now,
    $|B|$ is one more than it previously was, so this works. And note that this
    procedure can be repeated a total of $n-1$ times, so we end up with a maximum
    possible $m=n+1$. Because the problem only requires at max $m=n$, we are done. $\square$
\end{solution}
