% December 2021

\begin{problem}[ISL 2002 C1]
    Let $n$ be a positive integer. Each point $(x,y)$ in the plane, where $x$
    and $y$ are non-negative integers with $x+y<n$, is coloured red or blue,
    subject to the following condition: if a point $(x,y)$ is red, then so are
    all points $(x',y')$ with $x'\leq x$ and $y'\leq y$. Let $A$ be the number
    of ways to choose $n$ blue points with distinct $x$-coordinates, and let
    $B$ be the number of ways to choose $n$ blue points with distinct
    $y$-coordinates. Prove that $A=B$.
\end{problem}

\begin{solution}[Ritwin]
    Note that $A=\prod_{x=0}^{n-1}h(x)$ and $B=\prod_{y=0}^{n-1}w(y)$ where
    $h(c)$ denotes the number of blue points with $x=c$ and $w(c)$ similarly
    for the number of blue points where $y=c$.
    
    Consider taking a coloring of points like one in the statement, and
    ``extending'' it. We will show that:
    \begin{enumerate}[itemsep=-2pt]
      \item The multiset $\{\,h(x) \mid 0\leq x<n\,\}$ is equal to
      $\{\,w(y) \mid 0\leq y<n\,\}$. Note that this preserves $A=B$.
      \item We can get to any coloring with this extension
    \end{enumerate}
    
    \large Part 1: Preservation of $A=B$ \small
    
    Assume that we color the originally blue point $(x,y)$ to red, and that this
    recoloring preserves the condition where $(x',y')$ is red for all $x'\leq x$
    and $y'\leq y$ if $(x,y)$ is red.
    
    Now, we must have that
    \begin{itemize}
      \item $(x',y)$ is blue for all $x'>x$ and $x'+y<n$
      \item $(x,y')$ is blue for all $y'>y$ and $x+y'<n$
    \end{itemize}
    
    because otherwise, we would have had a red point either above or to the right
    of $(x,y)$, originally a blue point, which doesn't satisfy our original condition.
    
    Now, note that the number of points colored either red of blue with
    $x$-coordinate $x$ is $n-x$, and similarly the number of points of either
    color with $y$-coordinate $y$ is $n-y$. Thus, $h(x) = n-x-y$ and $w(y) = n-y-x$
    before the change. Note that $h(x)=w(y)=n-x-y$. After the change, each of
    $h(x)$ and $w(y)$ decreases by exactly one, so we still have $h(x)=w(y)=n-x-y+1$.
    Therefore, our products for $A$ and $B$ stay the same.
    
    Obviously, if no points are red, then we have $A=B$ by symmetry; just swap
    the $x$ and $y$-coordinates around.
    
    Therefore, if we can use our "extension" to get to every coloring that
    satisfies our original condition, then we have $A=B$ for all such colorings.
    
    \large Part 2: Getting to every coloring \small
    
    We will present an algorithm to start with no red points and add new red
    points such that our original condition is always satisfied, in such a way
    to get to the desired coloring.
    
    Sort the red points so that the value $x+y$ is nondecreasing for all points
    $P(x,y)$ in our ordering. For example, $\{(0,2),(1,0),(0,0),(0,1)\}$ would
    be sorted to become $[(0,0),(0,1),(1,0),(0,2)]$. Note that the ordering of
    points where $x+y=c$ for some fixed $c$ does not matter;
    $[(0,0),(1,0),(0,1),(0,2)]$ is also a valid ordering.
    
    Now, notice that this ordering of points to add will always preserve our
    original condition. When we add some point $(x,y)$, $(x-1,y)$ and $(x,y-1)$
    would both be earlier in our ordering, so if $(x-1,y)$ was colored, and
    colored red, then we would have already added it. The same is true for $(x,y-1)$.
    
    Therefore, this ordering of points is valid, so we can always construct any
    coloring of points by "extending" our current coloring and starting with no
    red points. Because each extension preserves $A=B$, we know that $A=B$ in
    our final coloring. $\square$
\end{solution}
