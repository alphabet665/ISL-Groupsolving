% May 2022

\begin{problem}[ISL 2006 N3]
    We define a sequence $(a_i)_{i=1}^\infty$ by
    \[a_n = \frac1n\bigg(\floor{\frac n1} + \floor{\frac n2} + \cdots + \floor{\frac nn}\bigg).\]
    \begin{enumerate}[(a),itemsep=-6pt]
        \item Prove that $a_{n+1}>a_n$ infinitely often.
        \item Prove that $a_{n+1}<a_n$ infinitely often.
    \end{enumerate}
\end{problem}

\begin{solution}[Ritwin]
    Note that \[a_n = \frac1n \sum_{k=1}^n d(k), \tag{1}\]
    where $d(n)$ is the number of divisors of $n$.
    
    Thus, at every $n$ such that $d(n) \geq d(k)$ for all $k \leq n$ (so $n$ is
    a superabundant number), (a) is satisfied. However note that if there were
    to be only finitely many examples of this, it would imply that $d(n)$ is
    bounded, which it obviously isn't (take the sequence $d(2^n) = n+1$ for example).
    Therefore $d(n) > a_{n-1} \implies a_n > a_{n-1}$ so (a) is done. $\square$
    
    Also note $a_n > 2$ for all $n \geq 6$. It is easy to see this, since every
    term in the sum in (1) is at least $2$ except $d(1) = 1$, so $na_n \geq 2n-1$.
    But also note $d(6) = 4$, so when $n \geq 6$, $na_n \geq 2n+1 \implies a_n > 2$.
    Thus, when $n$ is prime, $d(n) = 2$ and so obviously it decreases the average
    number of factors: $d(n) = 2 < a_{n-1} \implies a_n < a_{n-1}$. Thus (b) is
    done as well. $\square$
\end{solution}
