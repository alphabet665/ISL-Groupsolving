% January 2022

\begin{problem}[ISL 2020 N2]
    For each prime $p$, construct a graph $G_p$ on $\{1,2,\ldots p\}$, where
    $m \neq n$ are adjacent if and only if $p$ divides $(m^2+1-n)(n^2+1-m)$.
    Prove that $G_p$ is disconnected for infinitely many $p$.
\end{problem}

\begin{solution}[Ritwin]
    Note that we can reproduce the same edges by connecting $m \rightarrow m^2+1$.
    But if we have at least two values $m$ such that $m \equiv m^2+1 \pmod p$,
    then we will have at most $p-2$ edges and cannot connect all $p$ nodes
    together (which can be proven by considering that each edge reduces the
    number of connected components by either $0$ or $1$).
    
    Note that $m^2-m+1 = \tfrac{m^3+1}{m+1}$, so $m \equiv m^2+1 \pmod p$ is
    equivalent to \[(-m)^3 \equiv 1 \pmod p\] when $m \neq p-1$, so we need
    $1$ to have at least $3$ distinct cube roots modulo $p$.
    
    Consider $p = 3k+1$ (by Dirichlet's Theorem, there are infinitely many such
    primes). Then, $\varphi(p) = p-1 = 3k$ is a multiple of $3$. Gauss proved
    that if $p$ is a prime (amongst other cases), there exists a primitive root,
    call it $g$. Now, note that all of $g^0$, $g^k$, and $g^{2k}$ are cube roots
    of $1$ modulo $p$, so our proof is done. $\square$
\end{solution}
