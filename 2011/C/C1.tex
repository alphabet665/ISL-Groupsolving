% June 18, 2022

\begin{problem}[ISL 2011 C1]
    Let $n > 0$ be an integer. We are given a balance and $n$ weights
    of weight $2^0, 2^1, \cdots, 2^{n-1}$. We are to place each of the
    $n$ weights on the balance, one after another, in such a way that
    the right pan is never heavier than the left pan. At each step we
    choose one of the weights that has not yet been placed on the balance,
    and place it on either the left pan or the right pan, until all of
    the weights have been placed. \\
    Determine the number of ways in which this can be done.
\end{problem}

\begin{solution}
    Let $f(k)$ be the number of ways when $n=k$.

    We claim that \[f(n) = (2n-1)!! = \prod_{k=1}^n (2k-1).\]
    We prove via induction, with the base case $n=1$ being obvious.

    Now we consider when the $2^0$ weight is placed. If it is placed
    first, it can only go on the left pan, and otherwise it can go on
    either pan.

    After the $2^0$ weight is fixed, the other weights can go in all
    the $f(n-1)$ ways. This is because the only way it \textit{wouldn't}
    work is if the $2^0$ weight is placed on the right pan when the
    two pans have equal nonzero weight. However, the two pans having
    equal weight is impossible, which is easily proven by noting that
    \[2^k > 2^{k-1} + \cdots + 2^1 + 2^0 = 2^k-1,\]
    so whichever pan has the largest weight is always heavier.
    
    This yields $2n-1$ ways to put the $2^0$ weight, each of which
    has $f(n-1)$ ways to place the $n-1$ other weights, so
    \[(2n-1) \cdot f(n-1) = (2n-3)!! \cdot (2n-1) = (2n-1)!!\]
    as desired. $\square$
\end{solution}
